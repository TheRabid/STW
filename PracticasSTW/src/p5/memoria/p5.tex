\documentclass[a4paper]{article}

\usepackage[utf8]{inputenc}
\usepackage[spanish]{babel}
\usepackage{graphics}
\usepackage{caption}
\usepackage{subcaption}
\usepackage[demo]{graphicx}
\usepackage{enumitem}
\usepackage{longtable}
\usepackage{listings}
\usepackage{listingsutf8}
\usepackage{framed}
\usepackage{float}
\usepackage{hyperref}
\usepackage{anysize}
\usepackage{fancyhdr}
\usepackage{eurosym}

% para poner encabezado a la derecha
\pagestyle{fancy}
\fancyhf{}
\fancyhead[LE,RO]{Sistemas y Tecnologías Web}

\newcommand{\HRule}{\rule{\linewidth}{0.35mm}} % Defines a new command for the horizontal lines, change thickness here

\begin{document}

% % cambiar margenes
\marginsize{3cm}{3cm}{2.5cm}{2.5cm} 

\begin{flushright}
	\textsc{\huge Práctica 5\\}
	\textsc{\tiny " "\\}
	\textsc{\large
		Jaime Ruiz-Borau Vizárraga (546751)\\
	}
	\HRule \\
\end{flushright}

\section{Descripción del API desarrollada}
\setcounter{page}{1}
\subsection{Recursos identificados}
\paragraph{}El API desarrollada se fundamenta en dos recursos básicos del sistema: las \textbf{memos} (notas) y los \textbf{usuarios} del sistema.
\subsection{Operaciones implementadas sobre los recursos}
\paragraph{}A continuación, se listan las operaciones implementadas sobre los recursos identificados y su comportamiento al ser invocadas:
\paragraph{Memos}
\begin{itemize}
	\item \textbf{GET} del endpoint \textbf{/memo}: Devuelve todas las memos almacenadas en la base de datos con el formato de un \textit{identificador} y un \textit{enlace a la memo}, por cada una de las memos almacenadas.
	\item \textbf{POST} del endpoint \textbf{/memo}:  
	Almacena una nueva memo en la base de datos. Recibe los parámetros de la memo del \textit{body} de la petición.
	\item \textbf{DELETE} del endpoint \textbf{/memo}: Elimina todas las memos almacenadas en la base de datos.
	\item \textbf{GET} del endpoint \textbf{/memo/:id}: Devuelve toda la información de la memo referenciada por \textit{id}.
	\item \textbf{PUT} del endpoint \textbf{/memo/:id}: Actualiza la información de la memo referenciada por \textit{id}.
	\item \textbf{DELETE} del endpoint \textbf{/memo/:id}: Elimina la memo referenciada por \textit{id} de la base de datos.
\end{itemize}

\paragraph{Usuarios}
\begin{itemize}
	\item \textbf{GET} del endpoint \textbf{/user}: Devuelve todos los usuarios almacenados en la base de datos con el formato de un \textit{identificador} y un \textit{enlace al usuario}, por cada uno de los usuarios almacenados.
	\item \textbf{POST} del endpoint \textbf{/user}:  
	Almacena un nuevo usuario en la base de datos. Recibe los parámetros del usuario del \textit{body} de la petición.
	\item \textbf{DELETE} del endpoint \textbf{/user}: Elimina todos los usuarios almacenados en la base de datos.
	\item \textbf{GET} del endpoint \textbf{/user/:id}: Devuelve toda la información del usuario referenciada por \textit{id}.
	\item \textbf{PUT} del endpoint \textbf{/user/:id}: Actualiza la información del usuario referenciada por \textit{id}.
	\item \textbf{DELETE} del endpoint \textbf{/user/:id}: Elimina el usuario referenciado por \textit{id} de la base de datos.
\end{itemize}

\end{document}